\documentclass[sigconf]{acmart}

\usepackage{booktabs}
\usepackage[brazil]{babel}   
\usepackage[utf8]{inputenc}

% Copyright
%\setcopyright{none}
%\setcopyright{acmcopyright}
%\setcopyright{acmlicensed}
\setcopyright{rightsretained}
%\setcopyright{usgov}
%\setcopyright{usgovmixed}
%\setcopyright{cagov}
%\setcopyright{cagovmixed}

% DOI
\acmDOI{10.475/123_4}

% ISBN
\acmISBN{123-4567-24-567/08/06}

%Conference
\acmConference[WOODSTOCK'97]{ACM Woodstock conference}{July 1997}{El
  Paso, Texas USA}
\acmYear{1997}
\copyrightyear{2016}

\acmArticle{4}
\acmPrice{15.00}

\begin{document}
\title{Proposta de solução do Problema de Dominação de Rainhas Utilizando ILS}
\titlenote{Produces the permission block, and copyright information}

\author{Maria Edoarda Vallim Fonseca}
\affiliation{%
  \institution{Institute of Computing -- UFF}
  \city{Niterói}
  \country{Brazil}
}
\email{medoarda@id.uff.br}

\author{Thales Athayde Santos}
\affiliation{%
  \institution{Institute of Computing -- UFF}
  \city{Niterói}  
  \country{Brazil}
}
\email{thalesathaydesantos@id.uff.br}

\begin{abstract}
Text...
\end{abstract}

%
% The code below should be generated by the tool at
% http://dl.acm.org/ccs.cfm
% Please copy and paste the code instead of the example below.
%
\begin{CCSXML}
<ccs2012>
 <concept>
  <concept_id>10010520.10010553.10010562</concept_id>
  <concept_desc>Computer systems organization~Embedded systems</concept_desc>
  <concept_significance>500</concept_significance>
 </concept>
 <concept>
  <concept_id>10010520.10010575.10010755</concept_id>
  <concept_desc>Computer systems organization~Redundancy</concept_desc>
  <concept_significance>300</concept_significance>
 </concept>
 <concept>
  <concept_id>10010520.10010553.10010554</concept_id>
  <concept_desc>Computer systems organization~Robotics</concept_desc>
  <concept_significance>100</concept_significance>
 </concept>
 <concept>
  <concept_id>10003033.10003083.10003095</concept_id>
  <concept_desc>Networks~Network reliability</concept_desc>
  <concept_significance>100</concept_significance>
 </concept>
</ccs2012>
\end{CCSXML}

\ccsdesc[500]{Computer systems organization~Embedded systems}
\ccsdesc[300]{Computer systems organization~Redundancy}
\ccsdesc{Computer systems organization~Robotics}
\ccsdesc[100]{Networks~Network reliability}

\keywords{WORD1, WORD2, WORD3}

\maketitle

%%%%%%%%%%%%%%%%%%%%%%%%%%%%%%%%%%%%%%%%%%%%%%%%%%%%%%%%%%%%%%%%%%%%%%
\section{Introduction}

O problema de dominação de rainhas é um problema muito bem conhecido
dentre os problemas de xadrez. Nele, dado um tabuleiro nxn, 
queremos descobrir qual é a menor quantidade de rainhas possível de forma
à dominar todo o tabuleiro.

Nesse artigo, propomos uma solução baseada em ILS e iremos comparar
nossos resultados com o método descrito em~\cite{SimoneBeauvoir}, 
que utiliza Algoritmo Genético e é mais comum na literatura.

%%%%%%%%%%%%%%%%%%%%%%%%%%%%%%%%%%%%%%%%%%%%%%%%%%%%%%%%%%%%%%%%%%%%%%
\section{Bibliographic Review}

Textd.soso..

%%%%%%%%%%%%%%%%%%%%%%%%%%%%%%%%%%%%%%%%%%%%%%%%%%%%%%%%%%%%%%%%%%%%%%
\section{Methodology}

Textosoxc...d

%%%%%%%%%%%%%%%%%%%%%%%%%%%%%%%%%%%%%%%%%%%%%%%%%%%%%%%%%%%%%%%%%%%%%%
\section{Results}

Nossos testes foram rodados em uma máquina Intel Core i5-7200U com 8GB de RAM, usando o sistema operacional 
Manjaro Linux com o pacote gráfico KDE Plasma.

%%%%%%%%%%%%%%%%%%%%%%%%%%%%%%%%%%%%%%%%%%%%%%%%%%%%%%%%%%%%%%%%%%%%%%
\section{Conclusion}

Text...

\bibliographystyle{ACM-Reference-Format}
\bibliography{bibliography}

\end{document}
