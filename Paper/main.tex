\documentclass[sigconf]{acmart}

\usepackage{booktabs}
\usepackage[brazil]{babel}   
\usepackage[utf8]{inputenc}

% Copyright
%\setcopyright{none}
%\setcopyright{acmcopyright}
%\setcopyright{acmlicensed}
\setcopyright{rightsretained}
%\setcopyright{usgov}
%\setcopyright{usgovmixed}
%\setcopyright{cagov}
%\setcopyright{cagovmixed}

% DOI
\acmDOI{10.475/123_4}

% ISBN
\acmISBN{123-4567-24-567/08/06}

%Conference
\acmConference[WOODSTOCK'97]{ACM Woodstock conference}{July 1997}{El
  Paso, Texas USA}
\acmYear{1997}
\copyrightyear{2016}

\acmArticle{4}
\acmPrice{15.00}

\begin{document}
\title{Proposta de solução do Problema de Dominação de Rainhas Utilizando ILS}
\titlenote{Produces the permission block, and copyright information}

\author{Maria Edoarda Vallim Fonseca}
\affiliation{%
  \institution{Institute of Computing -- UFF}
  \city{Niterói}
  \country{Brazil}
}
\email{medoarda@id.uff.br}

\author{Thales Athayde Santos}
\affiliation{%
  \institution{Institute of Computing -- UFF}
  \city{Niterói}  
  \country{Brazil}
}
\email{thalesathaydesantos@id.uff.br}

\begin{abstract}
In this paper, we propose a solution to the Dominating Queens Problem
using Iterative Local Search and compare our results with a previous
proposed solution using Genetic Algorithm.
\end{abstract}

%
% The code below should be generated by the tool at
% http://dl.acm.org/ccs.cfm
% Please copy and paste the code instead of the example below.
%

\keywords{Dominating Queen Problem, ILS, Metaheuristic}

\maketitle

%%%%%%%%%%%%%%%%%%%%%%%%%%%%%%%%%%%%%%%%%%%%%%%%%%%%%%%%%%%%%%%%%%%%%%
\section{Introdução}

O problema de dominação de rainhas é muito bem conhecido
dentre os problemas de xadrez. Nele, dado um tabuleiro \textit{nxn}, temos n quadrados
dispostos nas linhas e n quadrados nas colunas. Quando uma rainha Q 
é disposta no tabuleiro, ela domina a linha, a coluna, e as diagonais
que passam pela sua posição. O objetivo do problema é descobrir a disposição da
 menor quantidade de rainhas possível de forma à dominar todo o tabuleiro.

 Algoritmos evolucionários provaram ter sucesso para resolver e otimizar
  uma grande variedade de problemas complexos, incluindo problemas combinatórios,
  como o estudado aqui, em um tempo computacional razoavelmente aceitável.~\cite{doerr2011evolutionary}

\textit{Local Search}, ou Busca Local, é um método heurístico para resolver problemas 
computacionalmente difíceis. Busca local pode ser usada em problemas
que possam ser formulados como achar a solução maximizando um critério entre várias
soluções possíveis. Algoritmos de busca local movem de solução à solução no espaço de 
soluções possíveis aplicando mudanças locais, até uma solução dita ótima ser encontrada.~\cite{hoos2004stochastic}

Um problema da Busca Local é que ela pode ficar presa em um mínimo local, sem conseguir melhorar
seu resultado. Para contornar esse problema, utiliza-se \textit{Iterative Local Search},
ou Busca Local Iterativa. Essa modificação consiste em iterar sobre chamadas da busca local, 
cada vez começando de um ponto diferente do conjunto de solução perturbando o mínimo local atual 
de modo que faça a solução chegar em outro ótimo local. Esta perturbação não pode ser muito forte nem 
muito fraca, pois corre o risco dela acabar encontrando o mesmo mínimo local ou servir como uma inicialização
aleatória.~\cite{lourencco2010iterated}

  Nesse artigo, propomos uma solução baseada em ILS e iremos comparar
nossos resultados com o método descrito em~\cite{alharbi2017genetic}, 
que utiliza Algoritmo Genético e é mais comum na literatura.

%%%%%%%%%%%%%%%%%%%%%%%%%%%%%%%%%%%%%%%%%%%%%%%%%%%%%%%%%%%%%%%%%%%%%%
\section{Motivação}

Decidimos escolher esse problema por ter sido um dos temas que já
havíamos abordado em uma das apresentações do trabalho, então já 
tínhamos um certo grau de familiaridade com o assunto. Pesquisando mais
à fundo, vimos que as soluções mais comuns para a solução do Problema 
de Dominação de Rainhas eram com \textit{Backtracking} e Algoritmo Genético.
Além disso, de acordo com~\cite{alharbi2017genetic}, existem muitos artigos procurando
os limites superior e inferior do problema, mas não existe muito esforço de pesquisa
na busca de soluções práticas para o problema.

Visto essas condições, decidimos propor uma solução baseada em \textit{Iterative
Local Search} para o problema e comparar os resultados com uma solução em
Algoritmo Guloso.

%%%%%%%%%%%%%%%%%%%%%%%%%%%%%%%%%%%%%%%%%%%%%%%%%%%%%%%%%%%%%%%%%%%%%%
\section{Methodology}

O \textit{Local Search} foi implementado com 1 de distância pela nossa implementação
estar utilizando uma matriz e não um vetor contendo todas as posições do tabuleiro. Embora usar um
\textit{Local Search} com distâncias maiores gere resultados melhores, isso teria a consequência
negativa de aumentar exponencialmente o tempo computacional da execução do algoritmo.

%%%%%%%%%%%%%%%%%%%%%%%%%%%%%%%%%%%%%%%%%%%%%%%%%%%%%%%%%%%%%%%%%%%%%%
\section{Resultados}

Nossos testes foram rodados em uma máquina Intel Core i5-7200U com 8GB de RAM, usando o sistema operacional 
Manjaro Linux com o pacote gráfico KDE Plasma. A linguagem de programação
utilizada foi Python 3.7.1.

%%%%%%%%%%%%%%%%%%%%%%%%%%%%%%%%%%%%%%%%%%%%%%%%%%%%%%%%%%%%%%%%%%%%%%
\section{Conclusão}

Text...

\bibliographystyle{ACM-Reference-Format}
\bibliography{bibliography}

\end{document}
