\Fn(\tcc*[h]{algorithm as a recursive function}){\FRecurs{some args}}{
  \KwData{Some input data\\these inputs can be displayed on several lines and one 
    input can be wider than line's width.}
  \KwResult{Same for output data}
  \tcc{this is a comment to tell you that we will now really start code}
  \If(\tcc*[h]{a simple if but with a comment on the same line}){this is true}{
    we do that, else nothing\;
    \tcc{we will include other if so you can see this is possible}
    \eIf{we agree that}{
      we do that\;
    }{
      else we will do a more complicated if using else if\;
      \uIf{this first condition is true}{
        we do that\;
      }
      \uElseIf{this other condition is true}{
        this is done\tcc*[r]{else if}
      }
      \Else{
        in other case, we do this\tcc*[r]{else}
      }
    }
  }
  \tcc{now loops}
  \For{\forcond}{
    a for loop\;
  }
  \While{$i<n$}{
    a while loop including a repeat--until loop\;
    \Repeat{this end condition}{
      do this things\;
    }
  }
  They are many other possibilities and customization possible that you have to
  discover by reading the documentation.
}
